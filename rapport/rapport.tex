\documentclass[a4paper]{article}
\title{TP1 : Ghostbusters\protect\\ Programmation temps-réel}
\author{Orphée Antoniadis \and Raed Abdennadher \and Steven Liatti}
\usepackage[francais]{babel}
\usepackage{fontspec}
\usepackage{amsmath}
\usepackage{amsfonts}
\usepackage{enumitem}
\usepackage{minted}
\usemintedstyle{colorful}
\setlength{\parindent}{0pt}
\usepackage[left=2.5cm,top=2.5cm,right=2.5cm,bottom=2.5cm]{geometry}

\begin{document}
\maketitle
\newpage

\section{Introduction}
Le but de ce travail était de réaliser un jeu qui est un mélange entre deux
anciens jeux vidéo : le Pacman et le Casse-brique. Il s’agit ici de faire
évoluer des « fantômes » sur une partie de l’écran de façon aléatoire et de
tenter d’en toucher un maximum avec une balle qui rebondit sur tout ce qu’elle
touche. Lorsque la balle touche un fantôme, le score du joueur augmente et le
fantôme touché disparaît. Les fantômes rebondissent aussi entre eux lorsqu’ils
se touchent (mais sans disparaître). En bas de l’écran, une raquette dirigée par
l’utilisateur permet de faire rebondir la balle pour lui éviter de « tomber ».
Dans ce dernier cas, celle-ci est perdue. 3 essais, c’est-à-dire 3 pertes de
balles sont admises. Au-delà, le jeu s’arrête et affiche un score équivalent
au nombre de fantômes touchés pendant la partie.

\section{Objectif pédagogique}
L'objectif de ce tp était d'apprendre à utiliser un RTOS et de comprendre son
fonctionnement. Pour se faire nous avions à disposition la carte Mylab2 accompagnée
de la librairie FreeRTOS contenant les primitives du RTOS ainsi que la librairie
myLab\_lib contenant des fonctions utilitaires pour communiquer avec les périphériques
de la carte.

\section{Structure du programme}
*faire un shéma bloc*

\section{Description des tâches}
	\subsection{Gestion du jeu}
	\inputminted[breaklines,breaksymbol=,linenos,frame=single,stepnumber=5,tabsize=2,firstline=53,lastline=64]{C}{../workspace/ghostbuster_02/src/game.c}
	Lorsque la partie commence, le joueur dois appuyer sur le joystick pour
	commencer le jeu et lancer la balle. C'est cette tâche qui s'occupe de la
	gestion de cette fonctionnalité. Pour détécter la pression du joystick, nous avons
	été obligé de passer par une attente "semi-passive". La tâche va vérifier toutes
	les 10ms l'état du joystick puis se mettre en attente passive et laisser la main
	aux autres tâches. Lorsque la joystick est appuyé, la tâche initialise le nombre
	de vies ainsi que le score du joueur, sort de sa boucle, débloque le sémaphore
	sur lequel attendait la tâche de la balle puis attend lui  même sur un autre
	sémaphore qui sera débloqué lorsque le joueur aura perdu la balle 3 fois.

	\subsection{Déplacement des fantômes}

	\subsection{Gestion de la balle}

	\subsection{Gestion de la raquette}

\end{document}
